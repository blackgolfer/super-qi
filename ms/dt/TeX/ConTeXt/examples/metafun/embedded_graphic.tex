\starttext
\startchapter[title={Embeded Graphics}]
In this chapter
we will introduce the interface between {\sl\CONTEXT}\ and {\sl \METAPOST}\ and demonstrate how the definitions of
the graphics can be embedded in the document source.

\startsection[title={External graphics}]

\stopsection

\startsection[title={Integrated graphics}]
\startsubsection[title={Primitive method}]
An integrated graphic is defined in the document source or in a style definition file. The most
primitive way of doing this is just inserting the code:

\starttyping
\startMPcode
fill fullcircle scaled 200pt withcolor .625white ;
\stopMPcode
\stoptyping

This will produce the following picture:

\startMPcode
fill fullcircle scaled 200pt withcolor .625white ;
\stopMPcode

Such a graphic is used once at the spot where it is defined.
\stopsubsection

\startsubsection[title={Usable graphics}]
A {\sl usable} graphic is calculated anew each time it is used. An example of defining a usable graphic:
\starttyping
\startuseMPgraphic{name}
fill fullcircle scaled 200pt withcolor .625yellow ;
\stopuseMPgraphic
\stoptyping
To use it, we place the the command
\starttyping
\useMPgraphic{name}
\stoptyping
at the spot to display the defined graphic.

\startuseMPgraphic{name}
fill fullcircle scaled 200pt withcolor .625yellow ;
\stopuseMPgraphic
\useMPgraphic{name}

Note that this graphic is calculated each time it is placed, which can be time consuming.

\stopsubsection
\startsubsection[title={Reusable graphics}]
For graphics that don't change, {\sl\CONTEXT}\ provides {\sl reusable} graphics:
\starttyping
\startreusableMPgraphic{name}
fill fullcircle scaled 200pt withcolor .625yellow;
\stopreusableMPgraphic
\stoptyping
This definition is accompanied by:
\starttyping
\reuseMPgraphic{name}
\stoptyping

\startreusableMPgraphic{name}
fill fullcircle scaled 200pt withcolor .625yellow;
\stopreusableMPgraphic
\reuseMPgraphic{name}


Imagine what happens when we add some buttons to an interactive document without
taking care of this side effect. All the frames would look different. Consider
the following example.

\startbuffer[a]
\startuniqueMPgraphic{right or wrong}
  pickup pencircle scaled .075 ;
  fill unitsquare withcolor .8white ;
  draw unitsquare withcolor .625red ;
  currentpicture := currentpicture
    xscaled OverlayWidth yscaled OverlayHeight ;
\stopuniqueMPgraphic
\stopbuffer

\typebuffer[a]

Let's define this graphic as a background to some buttons.

\startbuffer[b]
\defineoverlay[button][\uniqueMPgraphic{right or wrong}]
\setupbuttons[background=button,frame=off]
\stopbuffer

\startbuffer[c]
\hbox
  {\button {previous}          [previouspage]\quad
   \button {next}              [nextpage]\quad
   \button {index}             [index]\quad
   \button {table of contents} [content]}
\stopbuffer

\typebuffer[b,c]

The buttons will look like:

\startlinecorrection[blank]
\setupinteraction[state=start,color=,contrastcolor=]
\getbuffer[a,b,c]
\stoplinecorrection

Compare these with:

\startbuffer[a]
\startuniqueMPgraphic{wrong or right}
  pickup pencircle scaled 3pt ;
  path p ; p := unitsquare
    xscaled OverlayWidth yscaled OverlayHeight ;
  fill p withcolor .8white ;
  draw p withcolor .625red ;
\stopuniqueMPgraphic
\stopbuffer

\startlinecorrection[blank]
\getbuffer[a,b]
\defineoverlay[button][\uniqueMPgraphic{wrong or right}]
\setupinteraction[state=start,color=,contrastcolor=]
\getbuffer[c]
\stoplinecorrection

Here the graphic was defined as:

\typebuffer[a]

\stopsubsection
\stopsection
\stopchapter

\stoptext