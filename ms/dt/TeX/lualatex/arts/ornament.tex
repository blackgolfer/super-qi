\documentclass{minimal}
%\usepackage[left=1.75cm,right=1.75cm,top=2.5cm,bottom=2.5cm,paper=a4paper]{geometry}
\usepackage{tikz}
\usetikzlibrary{lindenmayersystems,shadings,positioning,shadows.blur}
\usepackage{calc}
\usepackage{pgfornament}
%\usepackage{fancyhdr}
% \pgfdeclarelayer{background}
% \pgfdeclarelayer{foreground}
% \pgfsetlayers{background,main,foreground}

\newcommand{\wb}[2]{\fontsize{#1}{#2}\usefont{U}{webo}{xl}{n}}
\newcommand{\wbc}[3]{\vspace*{#1}\begin{center}
    \wb{#2}{#2}#3\end{center}\vspace*{#1}}
\newlength{\wsp}\setlength{\wsp}{1ex}
\pgfdeclarelindenmayersystem{Hilbert curve}{
  \symbol{X}{\pgflsystemdrawforward}
  \symbol{+}{\pgflsystemturnright} % Explicitly define + and - symbols.
  \symbol{-}{\pgflsystemturnleft}
  \rule{A -> +BX-AXA-XB+}
  \rule{B -> -AX+BXB+XA-}
}
\newcommand{\ornamentshape}[3]{
    \node[shade,inner color=#1,outer color=#2, draw=#3,shading angle=45,rotate=45,
        blur shadow={shadow blur steps=5,shadow blur extra rounding=1.3pt}] at (O)
    %{\wbc{1ex}{8}{pq}}
    {\pgfornament[width=.6cm]{7}}
}
\newcommand{\hcshape}[3]{
    \shadedraw[inner color=#1,outer color=#2, draw=#3,
        blur shadow={shadow blur steps=5,shadow blur extra rounding=1.3pt},
        l-system={Hilbert curve, axiom=A, order=3, angle=90},
        rotate=45] l-system
}
\def\thisshape{\hcshape}
\begin{document}
\pagestyle{empty}
\begin{tikzpicture}[scale=1]
    \begin{scope} %Scope environment is used to limit the clip command
        \clip [rounded corners] (.5,.5) rectangle (7.5,7.5);
        %The base shape is repeatedly drawn using loop comands.
        \foreach \y in {0,.9,..., 9} % {0, 4.5*\shortd, ..., n*4.5*\shortd}
            {
                \foreach \x in {0,1.55884,..., 7.79423}
                    %{0, sqrt(3)*4.5*\shortd,..., upper limit}
                    {
                        %\coordinate (O) at (\x,\y);
                        [xshift=\x,yshift=\y]\hcshape{white}{black}{black};
                    }
            }
        \foreach \y in {-0.45,.45,..., 8.6}
            %{0-4.5*\shortd/2, 4.5*\shortd/2,..., n*4.5*\shortd}
            {
                \foreach \x in {0.779422,2.338267,..., 10}
                    %{0.5*sqrt(3)*4.5*\shortd, 1.5*sqrt(3)*4.5*\shortd/2,...,upper lim}
                    {
                        %\coordinate (O) at (\x,\y);
                        [xshift=\x,yshif=\y] \hcshape{white}{black}{black};
                    }
            }
    \end{scope}
\end{tikzpicture}
\end{document}