\documentclass{article}
\usepackage[pinyin,australian]{babel}

\babelprovide[main,import,language=Default]{australian}
\babelprovide[import,language=Chinese Simplified]{chinese-simplified}
\babelprovide[import,language=Chinese Traditional]{chinese-traditional}

\babelfont{rm}{Noto Serif}
\babelfont{sf}{Noto Sans}
\babelfont[chinese-simplified]{rm}{Noto Serif CJK SC}
\babelfont[chinese-simplified]{sf}{Noto Sans CJK SC}
\babelfont[chinese-traditional]{rm}{Noto Serif CJK TC}
\babelfont[chinese-traditional]{sf}{Noto Sans CJK TC}

\begin{document}

\section{Roman Family}

Australian English.    

\foreignlanguage{chinese-simplified}{汉语。}
\foreignlanguage{pinyin}{Pīnyīn.}
\renewcommand*{\familydefault}{\sfdefault}
\sffamily

\section{Sans Serif Family}

Australian English.

\foreignlanguage{chinese-simplified}{汉语。}
\foreignlanguage{pinyin}{Pīnyīn.}

\end{document}