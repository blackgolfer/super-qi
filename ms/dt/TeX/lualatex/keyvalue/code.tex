\documentclass[11pt
              ,a4paper
              ,deepblue={hexagon(fg=primaryc,bg=primary,img=c3)}
              ]{SuperQIBook}
\usepackage[base]{babel}
\usepackage{lipsum}
\definecolor[named]{primary}{HTML}{004040}
\definecolor[named]{secondary}{HTML}{000040}
\definecolor[named]{complement}{HTML}{400000}
\definecolor[named]{tertiary}{HTML}{404000}
\definecolor[named]{primaryc}{HTML}{ffffff}
\definecolor[named]{secondaryc}{HTML}{ffffff}
\definecolor[named]{complementc}{HTML}{ffffff}
\definecolor[named]{tertiaryc}{HTML}{ffffff}

%\clearpairofpagestyles
\titlehead{Topics in Mathematics }
\subject{Signal Processing }
\title{Adapted Wavelet Analysis from\\ Theory to Software }
\subtitle{}
\author{Mladen Victor Wickerhauser }
\date{1993 }
\publishers{A K Peters }
%theorem
\newcommand*{\sqbname}[1]{\textbf{\textcolor{complement}{#1}}}
\def\savelastnode{\pgfextra\edef\tmpA{\tikzlastnode}\endpgfextra}
\def\restorelastnode{\pgfextra\edef\tikzlastnode{\tmpA}\endpgfextra}
\tikzstyle{sqbbox}=[draw=secondary
                  ,rectangle
                  ,rounded corners
                  ,inner sep=10pt
                  ,inner ysep=10pt
                  %,text=secondaryc
                  ,text opacity=1
                  ,fill=tertiary
                  ,fill opacity=0.1
                  ]
\tikzstyle{fancytitle} =[fill=secondary
                        ,text=secondaryc
                        ,rounded corners
                        ,minimum width=.25\textwidth
                        ,inner ysep=4pt
                        ,inner xsep=4pt
                        ]
\tikzstyle{title}=[append after command={\savelastnode%
                   node[fancytitle,right=0pt,anchor=north west] at
                   (\tikzlastnode.north west) {#1}\restorelastnode}%
                  ]%
\usepackage{environ}
\usepackage{xparse}
\makeatletter
\newcounter{thm@Theorem}[chapter]
\newcounter{thm@Proposition}[chapter]
\newcounter{thm@Lemma}[chapter]
\newcounter{thm@Definition}[chapter]
\newcounter{thm@Corollary}[thm@Theorem]
\NewEnviron{qitheorem}[2][Theorem]{
\addtocounter{thm@#1}{1}
\begin{tikzpicture}
\IfNoValueTF{#2}{\def\theoremtitle{#1\ {\normalfont\thesection.\csname thethm@#1\endcsname.}}}
{\def\theoremtitle{#1\ {\normalfont\thesection.\csname thethm@#1\endcsname}: #2}}
\node [sqbbox,title=\theoremtitle] (box){%
\begin{minipage}{0.95\linewidth }
   \vfill
   \vskip 1.75em
   \emph{\BODY}
   \vfill
\end{minipage}
};
\end{tikzpicture}
}
\makeatother
%geoemtry
\usepackage{tkz-euclide}
%heperlink
\makeatletter
\usepackage[luatex
           ,unicode
           ,plainpages=false
           ,colorlinks=true
           ,pdfpagelabels
           ,urlcolor=tertiary
           ,filecolor=tertiary!75!black
           ,linkcolor=tertiary!75!black
           ,breaklinks
           ,hyperfootnotes=false
           ,bookmarks=true
           ,bookmarksopen=true
           ,linktocpage=true
           ,pdfsubject={\@subject}
           ,pdfauthor={\@author}
           ,pdftitle={\@title}
           %,pdfkeywords={tikz,pgf,pdf,graphique,euclide,geometry,points,line,circle,angle,arbelos,gold,ratio}
           %,pdfcreator={\sqbengine}
           ]{hyperref}
\makeatother
\usepackage{float}

%\tkzSetUpColors[background=black,text=white]
\def\pictureprimary{teal}
\def\picturesecondary{cyan}
\def\picturetertiary{purple}
\def\picturecomplement{orange}
\tkzSetUpPoint[size=2,color=\pictureprimary]
\tkzSetUpLine[ultra thin,color=\pictureprimary]
\tkzSetUpCompass[color=\picturecomplement,ultra thin,/tkzcompass/delta=10] 
\tikzset{label style/.append style={below,color=\pictureprimary,font=\scriptsize}}
\tikzset{new/.style={color=\picturecomplement,ultra thin}} 
\tikzset{step 1/.style={color=\picturesecondary,ultra thin}} 
\tikzset{step 2/.style={color=\picturetertiary,ultra thin}} 

\newtheorem{theorem}{Theorem}[section]
\newtheorem{corollary}{Corollary}[theorem]
\newtheorem{lemma}[theorem]{Lemma}

\newcommand{\newboxnote}{1}{
        \NewEnviron{#1}{1}{
                ##1
                \BODY
        }
}
\begin{document}
\sqbtitlepage
\frontmatter
\tableofcontents
\mainmatter
\pagestyle{scrheadings}
%\pagecolor{primaryc}
%\automark[chapter]{chapter}
%\automark*[section]{}
\chapter{Document Elements}
\section{Foot note}
\lipsum{1}\footnote{test}

\section{Theorems}
Theorems can easily be defined:

\begin{theorem}
        Let \(f\) be a function whose derivative exists in every point, then \(f\) is
        a continuous function.
\end{theorem}

\begin{theorem}[Pythagorean theorem]
        \label{pythagorean}
        This is a theorem about right triangles and can be summarised in the next
        equation
        \[ x^2 + y^2 = z^2 \]
\end{theorem}

And a consequence of theorem \ref{pythagorean} is the statement in the next
corollary.

\begin{corollary}
        There's no right rectangle whose sides measure 3cm, 4cm, and 6cm.
\end{corollary}

You can reference theorems such as \ref{pythagorean} when a label is assigned.

\begin{lemma}
        Given two line segments whose lengths are \(a\) and \(b\) respectively there is a
        real number \(r\) such that \(b=ra\).
\end{lemma}

\section{QI Theorems}

\begin{qitheorem}{Golden Ratio}
        In mathematics, expressed algebraically, for quantities $a$ and $b$ with $a>b>0$,
        \begin{equation*}
                \frac{a+b}{a} = \frac{a}{b} = \varphi
        \end{equation*}
        where the Greek letter $\varphi$  represents the \sqbname{golden ratio} or \sqbname{devine proportion}.
        In other words, two quantities are in the golden ratio if their ratio is the same as the ratio of their
        sum to the larger of the two quantities. [wikipedia]
\end{qitheorem}

\section{Figures}
\begin{figure}[H]
        \centering
        \centering
        \begin{tikzpicture}[scale=1]
                \tkzDefPoints{0/0/A,10/0/B}
                \tkzDefGoldenRatio(A,B)                  \tkzGetPoint{C}
                \tkzDefMidPoint(A,B)                     \tkzGetPoint{O_0}
                \tkzDefMidPoint(A,C)                     \tkzGetPoint{O_1}
                \tkzDefMidPoint(C,B)                     \tkzGetPoint{O_2}
                \begin{scope}[local bounding box = graph]
                        \tkzDrawSemiCircles[color=black](O_0,B)
                        \tkzDrawSegment(A,B)
                        \tkzDrawPoints(A,B,C,O_1,O_0,O_2)
                        \tkzLabelPoints[font=\footnotesize](A,B,C,O_1,O_0,O_2)
                \end{scope}
                \useasboundingbox (graph.south west) rectangle (graph.north east);
                \tkzLabelCircle[below=6pt,font=\scriptsize](O_1,C)(90){$\beta$}
                \tkzLabelCircle[below=6pt,font=\scriptsize](O_2,B)(90){$\gamma$}
                \tkzClipCircle[out](O_1,C)\tkzClipCircle[out](O_2,B)
                \tkzDrawSemiCircles[draw=none,fill=\pictureprimary!15!white](O_0,B)
                \tkzDrawSemiCircles[color=black](O_1,C O_2,B)
                \tkzLabelCircle[below=6pt,font=\scriptsize](O_0,B)(90){$\alpha$}
        \end{tikzpicture}
        \caption{Arbelos}
\end{figure}
\begin{qitheorem}[Lemma]{Arbelos Definition by Archimedes' Book of Lemmas}
        If\: $[AB]$ be the diameter of a semicircle and C any point on $[AB]$, and if semicircles
        be described within the first semicircle and having $[AC]$, $[CB]$  as diameters respectively,
        the figure included between the circumferences of the three semicircles is called \sqbname{arbelos}.
        Let $\beta, \gamma, \alpha$ be the semicircles of respectives diameters $[AC]$, $[CB]$ and $[AB]$,
        then the arbelos bounded by the three semicircles is denoted $(\beta,\gamma,\alpha)$ or $(A,C,B)$.
\end{qitheorem}
\chapter{Font Test}
\section{Another lipsum}
\lipsum{11-20}
\lipsum{21-25}
\lipsum{26-30}
\end{document}