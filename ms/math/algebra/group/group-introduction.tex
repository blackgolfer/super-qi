% language=us runpath=texruns:algebra/group

\startcomponent group-introduction

\environment group-environment

\startchapter[title={Introduction}]

\startsection[title={Algebraic laws}]
\regularLettrine{A}{large part} of algebra is conncerned with systems of elements which, like numbers, may be
combined by addition or multiplication or both. We are given a system whose elements are designated by letters
$a,b,c,\cdots$. We write $S=S(a,b,c,\cdots)$ for this system. The properties of these systems depend upon which
of the following basic laws hold:
{\setuptables[bodyfont=8pt,distance=small]
\placetable[here][tab:laws]{\small Algebraic laws.}
\starttable[|l|l|l|]
\HL
\NC \bf Laws \NC \bf Addition \NC \bf Multiplication \NC\SR
\HL
\NC Closure laws \NC $A_0$: Addition is well defined \NC $M_0$: Multiplication is well defined \NC\FR
\NC Associative laws \NC $A_1$: $(a+b)+c=a+(b+c)$ \NC $M_1$: $(ab)c=a(bc)$ \NC\MR
\NC Commutative laws \NC $A_2$: $a+b=b+a$ \NC $M_2$: $ba=ab$ \NC\MR
\NC Zero and unit\NC $A_3$: $\exists 0,\ni 0+a=a+0=a,\forall a$\NC $M_3$:$\exists 1,\ni 1a=a1=a,\forall a$\NC\MR
\NC Negatives and Inverses\NC $A_4$: $\forall a,\exists -a,\ni (-a)+a=a+(-a)=0$\NC $M_4$:
$\forall a\neq 0,\exists a^{-1},\ni (a^{-1})a=a(a^{-1})=1$\NC\MR
\NC \LOW{Distributive laws} \NC \TWO $D_1$: $a(b+c)=ab+ac$ \NC\LR
\NC \NC \TWO $D_2$: $(b+c)a=ba+ca$ \NC\LR
\HL
\stoptable}

Note: $A_0$ means that, for every ordered pair of elements $a,b\in S$, $a+b=c$ exists and is a unique element of
$S$,. Also, $M_0$ means that, $ab=d$ exists and is a unique element of $S$.
\stopsection

\startdefinition[dfn:fieldnring]{Field and Ring}
A system satisfying all these laws is called a field. A system satisfying $A_0$, $A_1$, $A_2$, $A_3$,
$A_4$, $M_0$, $M_1$ and $D_1$, $D_2$ is called a ring.
\stopdefinition

The parallelism between addition $A_0-A_4$ and multiplication $M_0-M_4$ is exploited in the use of logarithms,
where the basic correspondence between them is given by the law:
$$\log(xy)=\log(x)+\log(y).$$

In general an $n$-ary operation in a set $S$ is a function $f=f(a_1,\dots,a_n)$ of $n$ arguments
$(a_1,\dots,a_n)$ which are the elements of $S$ and whose values $f(a_1,\dots,a_n)=b$ is a unique element of
$S$ when $f$ is defined for these arguments. If, for every choice of $a_1,\dots,a_n$ in $S$, $f(a_1,\dots,a_n)$
is defined, we say that the operation $f$ is {\sl well defined} or that the set $S$ is {\sl closed}
with respect to the operation $f$.

In a field $F$, the addition and multiplication are well-defined binary operations, while the inversion
operation $f(a)=a^{-1}$ is a unary operation defined for every element except zero.

\startsection[sec:mappings]{Mappings}
\regularLettrine{A}{ very fundamental concept} of modern mathematics in that of a {\sl mapping} of a set $S$ 
into a set $T$.
\vskip 5ex
\startdefinition[dfn:mapping]{Mapping}
A {\sl mapping} $\alpha$ of a set $S$ into a set $T$ is a rule which assigns to each $x$ of the set $S$ a unique
$y$ of the set $T$.
\stopdefinition
Symbolically we wrtie this either of the notations:
$$ \alpha: x\longrightarrow y\quad\rm{or}\quad y=(x)\alpha.$$
The element $y$ is called the {\sl image} of $x$ under $\alpha$. If every $y$ of the set $T$ is the image of
at least one $x$ in $S$, we say that $\alpha$ is a mapping of $S$ onto $T$.

The mapping of a set into (or onto) itself are of particular importance. For example a rotation in a plane
may be regarded as a mapping of the set of points in the plance onto itself. Two mappings $\alpha$ and $\beta$
of a set $S$ into iteself may be combined to yield a third mapping of $S$ into itself, according to the
following definition.

\startdefinition[dfn:mapping_composition]{Product of Mappings}
Given two mappings $\alpha$, $\beta$, of a set $S$ into itself, we define a third mapping $\gamma$ of $S$
into itself by the rule: If $y=(x)\alpha$ and $z=(y)\beta$, then $z=(x)\gamma$. The mapping $\gamma$ is called
the product of $\alpha$ and $\beta$, and we write $\gamma=\alpha\beta$.
\stopdefinition

Here, since $y=(x)\alpha$ is unique and $z=(y)\beta$ is unique, $z=[(x)\alpha]\beta=(x)\gamma$ is defined for
every $x$ of $S$ and is a unique element of $S$.

\starttheorem[thm:product_mappings]{}
The mappings of a set $S$ into itself stisfy $M_0$, $M_1$, and $M_3$ if multiplication is interpreted to be
the product of mappings.
\stoptheorem
{\sl Proof:} It has already been noted that $M_0$ is satisfied. Let us consider $M_1$. Let $\alpha$,$\beta$,
$\gamma$ be three given mappings. Take any element $x$ of $S$ and let $y=(x)\alpha$, $z=(y)\beta$ and
$w=(x)\gamma$. Then $(x)[(\alpha\beta)\gamma]=z(\gamma)=w$, and $(x)[\alpha(\beta\gamma)]=y(\beta\gamma)=w$.
Since both mappings, $(\alpha\beta)\gamma$ and $\alpha(\beta\gamma)$, give the same image for every $x$ in $S$,
it follows that $(\alpha\beta)\gamma=\alpha(\beta\gamma)$.

As for $M_3$, let $1$ be the mapping such that $(x)1=x$ for every $x$ in $S$. Then $1$ is a unit in the sense
that for every mapping $\alpha$, $\alpha 1=1\alpha=\alpha$.

In general, neither $M_2$ nor $M_4$ holds for mappings. But $M_4$ holds for an important class of mappings,
namely, the one-to-one mappings of $S$ onto itself.

\startdefinition[dfn:one_to_one_mappings]{One-to-one Mappings}
A mapping $\alpha$ of a set $S$ onto $T$ is said to be one-to-one (which we will frequently write $1-1$)
if every element of $T$ is the image of exactly one element of $S$. We indicate such a mapping by the
notation: $\alpha: x\leftrightarrows y$, where $x$ is an element of $S$ and $y$ is an element of $T$.
We say that $S$ and $T$ have the same cardinal number of elments.
\stopdefinition

\starttheorem[thm:one_to_one_mappings]{}
The one-to-one mappings of a set $S$ onto itself satisfy $M_0$, $M_1$, $M_3$ and $M_4$.
\stoptheorem
{\sl Proof:} Since \in{theorem}[thm:product_mappings] covers $M_0$, $M_1$ and $M_3$, we need only veryfy
$M_4$. If $alpha: x\leftrightarrows y$ is a one-to-one mapping of $S$ onto itself, then by definition,
for every $y$ of $S$ there is exactly one $x$ of $S$ such that $y=(x)\alpha$. This assignment of a unique $x$
to each $y$ determines a one-to-one mapping $\tau: y\leftrightarrows x$ of $S$ onto itself. From the definition
of $\tau$ we see that $(x)(\alpha\tau)=x$ for every $x$ in $S$ and $y(\tau\alpha)=y$ for every $y$ in $S$.
Hence, $\alpha\tau=\tau\alpha=1$, and $\tau$ is a mapping satisfying the requirements for $\alpha^{-1}$ in
$M_4$.

We call a one-to-one mapping of a set onto itself a {\sl permutation}. When the given set is finite,
a permutation may be written by putting the elements of the set in a row and their images below them. Thus
$\alpha=\left(
    \startmathmatrix[n=3,align={left,left,left}]
    \NC 1 \NC 2 \NC 3 \NR
    \NC 2 \NC 3 \NC 1 \NR
    \stopmathmatrix
    \right)
$
and
$\beta=\left(
    \startmathmatrix[n=3,align={left,left,left}]
    \NC 1 \NC 2 \NC 3 \NR
    \NC 1 \NC 3 \NC 2 \NR
    \stopmathmatrix
    \right)
$
are two permutations of the set $S(1,2,3)$. Their product is defined to be the permutation
$\alpha\beta=\left(
    \startmathmatrix[n=3,align={left,left,left}]
    \NC 1 \NC 2 \NC 3 \NR
    \NC 3 \NC 2 \NC 1 \NR
    \stopmathmatrix
    \right)$.
Note that the product rule for permutations given here is obtained by multiplying from left to right.
Some authors define the product so that multiplication is from right to left.

\stopsection

\startsection[title={Definitions for groups and some related systems}]
\regularLettrine{W}{e} see that, as single operations, the laws governing addition and multiplication are the
same. Of these, all but the commutative law are satisfied by the product rule for the one-to-one mappings of
a set onto itself. The law obeyed by these one-to-one mappings are those which we shall use to define a group.

\startdefinition[dfn:group1]{First Definition of a Group}
A group $G$ is a set of elements $G(a,b,c,\dots)$ and a binary operation call ``product'' such that:
%\definesymbol[head][$G_{\currentitemnumber}$]
%\startitemize[head]
\startitemize
\txt{$G_0$. Closure Law.}\ For every ordered pair $a$, $b$ of elements of $G$, the product $ab=c$ exists and is
    a unique element of $G$.
\txt{$G_1$. Associative Law.}\ $(ab)c=a(bc)$.
\txt{$G_2$. Existence of Unit.}\ An element $1$ exists such that $1a=a1=a$ for every $a$ of $G$.
\txt{$G_3$. Existence of Inverse.}\ For every $a$ of $G$ there exists an element $a^{-1}$ of $G$ such that
    $a^{-1}a=aa^{-1}=1$.
\stopitemize
\stopdefinition
These laws are redundant. We may omit one-half of $G_2$ and $G_3$, and replace them by:
\startitemize
\txt{$G_2^*$.}\ {\sl An element $1$ exists such that $1a=a$ for every $a$ of $G$.}
\txt{$G_3^*$.}\ {\sl For every $a$ of $G$ there exists an element $x$ of $G$ such that $xa=1$.}
\stopitemize
We can show that these in turn imply $G_2$ and $G_3$. For a given $a$ let
$$xa=1\quad\rm{and}\quad yx=1$$
by $G_3^*$. Then we have
$$ax=1(ax)=(yx)(ax)=y[x(ax)]=y[1x]=yx=1,$$
so that $G_3$ is satisfied. Also,
$$a=1a=(ax)a=a(xa)=a1,$$
so that $G_2$ is satisfied.

The uniqueness of the unit $1$ and of an inverse $a^{-1}$ are readily established (see Ex. 13). We could,
of course, also replace $G_2$ and $G_3$ by the assumption of the existence of $1$ and $x$ such that:
$a1=a$ and $ax=1$. But if we assume that they satisfy $a1=a$ and $xa=1$, the situation is slightly different.

There are a number of ways of bracketing an ordered sequency $a_1a_2\cdots a_n$ to give it a value by
calculating a succession of binary products. For $n=3$ there are just two ways of bracketing, namely,
$(a_1a_2)a_3$ and $a_1(a_2a_3)$, and the associative law asserts the equality of these two products.
An important consequence of the associative law is the {\sl gneralized associative law}.

{\sl All ways of bracketing an ordered sequence $a_1a_2\cdots a_n$ to give it a value by calculating
a succession of binary products yield the same value}.

It is a simple matter, using induction on $n$, to prove that the generalized associative law is a consequence
of the associative law (see Ex. 1).

Another definition may be given which does not explicitly postulate the existence of the unit.


\stopsection

\stopchapter

\stopcomponent