% language=us runpath=texruns:algebra/group

\startcomponent group-preface

\environment group-environment

\starttitle[title={Preface}]
\regularLettrine{T}{he} present volume is intended to serve a dual purpose. The first ten chapters are meant to be the basis for a
course in Group Theory, and exercises have been included at the end of each of these chapters. The last ten
chapters are meant to be useful as optional material in a course or as reference material. When used as a text,
the book is intended for students who have had an introductory course in Modern Algebra comparable to a
course taught from Birkhoff and MacLane's {\sl A Survey of Modern Algebra}. I have tried to make this book
as self-contained as possible, but where background material is needed references have been given, chiefly
to Birkhoff and MacLane.

Current research in Group Theory, as witnessed by the publications covered in {\sl Mathematical Reviews}
is vigorous and extensive. It is no longer possible to cover the whole subject matter or even to give a
complete bibliography. I have therefore been guided to a considerable extent by my own interests in selecting
the subjects treated, and the bibliography covers only references made in the book itself. I have made a
deliberate effort to curtail the treatment of some jubjects of great interest whose detailed study is
readily available in recent publications. For detailed investigations of infinite Abelian groups, the reader
is referred to the appropriate sections of the second edition of Kurosch's {\sl Theory of Groups} and
Kaplansky's monograph {\sl Infinite Abelian Groups}. The monographs {\sl Structure of a Group and the
Structure of its Lattice of Subgroups} by Suzuki and {\sl Generators and Relations for Discrete Groups} by
Coxeter and Moser, both in the Ergebnisse series, are recommended to the reader who whishes to go further
with these subjects.

This book developed from lecture notes on the course in Group Theory which I have given at The Ohio State
University over a period of years. The major part of this volume in its present form was written at Trinity
College, Cambridge, during 1956 while I held a Fellowship from the John Simon Guggenheim Foundation.
I give my thanks to the Foundation for the grant enabling me to carry out this work and to the Fellows of
Trinity College for giving me the privileges of the College.

I must chiefly give my thanks to Professor Philip Hall of King's College, Cambridge, who gave me many valuable
suggestions on the preparation of my manuscript and some unpublished material of his own. In recognition
of his many kindnesses, this book is dedicated to him.

I wish also to acknowledge the helpfulness of Professors Herbert J. Ryser and Jan Korringa and also Dr. Ernest T.
Parker in giving me their assistance on a number of matters relating to the manuscript.

\blank[big,samepage]

\startlines
Marshall Hall, Jr.
Columbus, Ohio
%\currentdate[month,year]
\stoplines

\stoptitle

\stopcomponent