\def\plateName{plate}

\define[2]\bookTitle{\begingroup\lefttoright\vbox{\vskip -\topspace\hskip0pt\hfill\hbox{\hskip-1pt\putLettrine{\plateName}\hskip-1pt}\hfill\hskip0pt\vskip 5\baselineskip~\hfill\hbox{\tfd #2}\hfill~}\endgroup}

\newdimen\LettrineSize
\newcount\LettrineScale
\newdimen\LettrinePadding
\newdimen\PlateWidth
\newdimen\PlateHeight
\newdimen\pictureIndent
\newdimen\PcH
\LettrineSize=1.5cm
\LettrineScale=300
\LettrinePadding=2mm
\PlateHeight=1.5cm
\PlateWidth=7.6cm
\def\LettrineTotalWidth{\dimexpr \LettrineSize + \LettrinePadding \relax}
\def\LettrineTotalHeight{\dimexpr \LettrineSize + \LettrinePadding \relax}
\def\LettrineTotalHeightMinusLine{\dimexpr \LettrineTotalHeight -0.8\baselineskip}
\def\defaultLettrineLines{4}
\pictureIndent=0.5\textwidth
\newdimen\oddpict
\oddpict=\backspace
\advance \oddpict by \textwidth
\advance \oddpict by -\pictureIndent

\newdimen\evenpict
\evenpict=\cutspace
\advance \evenpict by \pictureIndent

\define\drawCurrentPictureWide{
    \setbox1\vbox{\drawCurrentPicture}
    \newdimen\pictVOffset
    \pictVOffset=\topmargin
    \advance\pictVOffset by \pagetotal
    \pictureIndent=\wd1
    \PcH=\ht1
    \advance\PcH by \pagetotal
    \advance \pictureIndent by 1cm
    \advance \pictureIndent by -\cutspace
    \doifoddpageelse{
        \oddpict=\backspace
        \advance \oddpict by \makeupwidth
        \advance \oddpict by -\pictureIndent
        \definelayer[curpic]
        [x=\oddpict,y=\pictVOffset,width=\textwidth,height=\textheight]
    }{
        \evenpict=\cutspace
        \advance \evenpict by \pictureIndent
        \definelayer[curpic]
        [x=0cm,y=\pictVOffset,width=\evenpict,height=\textheight]
    }
    \setlayer[curpic][voffset=0.5cm]{\hskip 0.5cm\box1\hskip 0.5cm}
    \setupbackgrounds[page][background=curpic]
}

% Often it's convenient to skip come lines and continue below main figure, 
% leaving all the indentation stuff above
\newdimen\dTB
\def\skipToPictureBottom{
    \dTB=\PcH
    \advance\dTB by -\pagetotal
    \vskip \dTB
    \resetInitialIndentation
}

\newdimen\pictureWithLettrineLineLength
\newdimen\pictureWithLettrineIndent
\newdimen\pictureOnlyLineLength
\newdimen\pictureWithLettrineLineLength
\newcount\Y
\newcount\Yy
\newcount\X
\X=0
\newcount\numOfIndLines

\newif\ifinsideII
\insideIIfalse

\def\resetInitialIndentation{\parshape=0\global\numOfIndLines=0\global\def\par{\endgraf}\everypar=\expandafter{\the\toks0}\global\insideIIfalse}

\newcount\linesofar

\toks0=\expandafter{\the\everypar}

\def\initialIndentation{\dosingleempty\initialindentation}
\def\initialindentation[#1]#2{
\global\toks0=\expandafter{\the\everypar}
\iffirstargument
    \def\lettrineLines{#1}
\else%
    \def\lettrineLines{\defaultLettrineLines}
\fi
\ifinsideII\resetInitialIndentation\fi
\global\insideIItrue
\ \vskip 0pt
%https://books.google.com/books?id=iDb2BwAAQBAJ&pg=PA107&lpg=PA107&dq=tex+parshape+paragraphs&source=bl&ots=GmFvQnIUr7&sig=oJ-mtVEwOttepWpGi2BABemAMWw&hl=en&sa=X&ved=0ahUKEwjxrbPk-uXKAhVClB4KHSxUBqwQ6AEIQTAF#v=onepage&q=tex%20parshape%20paragraphs&f=false
\linesofar=0
\global\numOfIndLines=#2
\def\par{{\endgraf\global\linesofar=\prevgraf}}
\everypar={\ifnum\linesofar<\numOfIndLines\prevgraf=\linesofar\else\resetInitialIndentation\fi} %
\pictureOnlyLineLength=\textwidth
\pictureWithLettrineLineLength=\textwidth
\advance \pictureOnlyLineLength by -\pictureIndent
\doifoddpageelse{
    \pictureWithLettrineLineLength=\pictureOnlyLineLength
    \advance \pictureWithLettrineLineLength by -\LettrineTotalWidth
    \def\partindent{0cm \pictureOnlyLineLength}
    \ifnum\textdirection=0
        \def\doubleindent{\LettrineTotalWidth \pictureWithLettrineLineLength}
    \fi
    \ifnum\textdirection=1
        \def\doubleindent{0cm \pictureWithLettrineLineLength}
    \fi
}{
    \pictureWithLettrineIndent=\pictureIndent
    \advance \pictureWithLettrineIndent by \LettrineTotalWidth
    \pictureWithLettrineLineLength=\pictureOnlyLineLength
    \advance \pictureWithLettrineLineLength by -\LettrineTotalWidth
    \def\partindent{\pictureIndent \pictureOnlyLineLength}
    \ifnum\textdirection=0
        \def\doubleindent{\pictureWithLettrineIndent \pictureWithLettrineLineLength}
    \fi
    \ifnum\textdirection=1
        \def\doubleindent{0cm \pictureWithLettrineLineLength}
    \fi
}
\def\preinitindent{}
\Yy=\lettrineLines
\advance \Yy by 1
\Y=\lettrineLines
\loop
\expandafter\def\expandafter\preinitindent\expandafter{\preinitindent\doubleindent}
\advance \Y by -1
\ifnum \Y>0
\repeat
\def\initindent{\preinitindent}
\global\X=#2
\ifnum \X>\Yy
\advance \X by -\lettrineLines
\loop
\expandafter\def\expandafter\initindent\expandafter{\initindent\partindent}
\advance \X by -1
\ifnum \X>1
    \repeat
\fi
\parshape=#2
\initindent
0cm \textwidth
}

\newwrite \lettrineslist
\openout \lettrineslist = lettrines/lettrineslist.txt\relax

\def\putzero{0}

\def\putLettrine#1{%
    \write \lettrineslist{#1}%
    \expandafter\ifx\csname ltrn#1\endcsname\relax
        \global\expandafter\newcount\csname ltrn#1\endcsname\relax
        \global\csname ltrn#1\endcsname 0
    \else
        \global\expandafter\advance\csname ltrn#1\endcsname by 1
    \fi
    \doifelsefile{lettrines/#1\putzero.mps}
    {\textdirection 0 \doifelsefile{lettrines/#1\the\csname ltrn#1\endcsname.mps}
        {\ifx#1\plateName%
                \externalfigure[lettrines/#1\the\csname ltrn#1\endcsname.mps][width=\PlateWidth]%
            \else%
                \externalfigure[lettrines/#1\the\csname ltrn#1\endcsname.mps][height=\LettrineSize]%
            \fi}
        {\ifx#1\plateName%
                \externalfigure[lettrines/#1\putzero.mps][width=\PlateWidth]%
            \else%
                \externalfigure[lettrines/#1\putzero.mps][height=\LettrineSize]%
            \fi}}
    {\doifelsefile{lettrines/#1.svg}
        {\externalfigure[lettrines/#1.svg][height=\LettrineSize]%
        }
        {\doifelsefile{lettrines/#1.pdf}
            {\externalfigure[lettrines/#1.pdf][width=\LettrineSize,height=\LettrineSize]}
            {\ifx#1\plateName%
                    \startMPcode
                    input byrne.mp;
                    path plateBG, plateFrame;
                    plateBG := (reverse(fullsquare) xscaled (\the\PlateWidth)) yscaled (\the\PlateHeight);
                    plateFrame := (reverse(fullsquare) xscaled (\the\PlateWidth - 2pt)) yscaled (\the\PlateHeight -2pt);
                    fill plateBG withcolor orange;
                    draw plateFrame withcolor white;
                    string pictureContents;
                    pictureContents := "path givenOutline[];";
                    pictureContents := pictureContents & "givenOutline0 := " & pathToString(plateBG) & "; ";
                    pictureContents := pictureContents & "givenOutline1 := " & pathToString(plateFrame) & "; ";
                    pictureContents := pictureContents & "givenOutline2 := " & pathToString((0,0)) & "; ";
                    pictureContents := pictureContents & "numeric totalLines; totalLines := 2;";
                    write pictureContents to "lettrines/lettrine_#1.mp";
                    write EOF to "lettrines/lettrine_#1.mp";
                    \stopMPcode%
                \else%
                    \startMPcode
                    input byrne.mp;
                    picture lettrineOutline, lettrinePicture;
                    path lettrineFrame, lettrineBG;
                    lettrineBG := reverse(fullsquare) scaled (\the\LettrineSize);
                    lettrineFrame := reverse(fullsquare) scaled (\the\LettrineSize - 2pt);
                    fill lettrineBG withcolor orange;
                    draw lettrineFrame withcolor white;
                    lettrineOutline := outlinetext("#1") scaled \the\LettrineScale/100;
                    lettrineOutline := lettrineOutline shifted -1/2[ulcorner(lettrineOutline), lrcorner(lettrineOutline)];
                    lettrinePicture := textext("#1") scaled \the\LettrineScale/100;
                    lettrinePicture := lettrinePicture shifted -1/2[ulcorner(lettrinePicture), lrcorner(lettrinePicture)];
                    draw lettrinePicture withcolor white;
                    string pictureContents;
                    pictureContents := "path givenOutline[];";
                    pictureContents := pictureContents & "givenOutline0 := " & pathToString(lettrineBG) & "; ";
                    pictureContents := pictureContents & "givenOutline1 := " & pathToString(lettrineFrame) & "; ";
                    j := 1;
                    for i within lettrineOutline:
                    j := j + 1;
                    pictureContents := pictureContents & "givenOutline" & decimal(j) & " := " & pathToString(pathpart i) & "; ";
                    endfor;
                    pictureContents := pictureContents & "numeric totalLines; totalLines := " & decimal(j) & ";";
                    write pictureContents to "lettrines/lettrine_#1.mp";
                    write EOF to "lettrines/lettrine_#1.mp";
                    \stopMPcode%
                \fi}}}}

% This thing removes space between an initial and the letter after it, if no space is needed
\startluacode
formatting = formatting or {}
function formatting.dedent (str)
if string.sub(str, 0, 1) ==  " " then
tex.print(str)
else
tex.print("\\,\\hskip -\\LettrinePadding " .. str)
end
end
\stopluacode

\def\regularLettrine#1#2{\placefigure[left,none]{}{\vskip -5pt\hbox{\putLettrine{#1}\hskip-4pt}\vskip -10pt}\noindent{\sc\ctxlua{formatting.dedent([==[#2]==])}}}
